\documentclass{beamer}

\usepackage[T2A]{fontenc}
\usepackage[utf8]{inputenc}
\usepackage[english,russian]{babel}
\usepackage{amssymb,amsfonts,amsmath,mathtext}
\usepackage{cite,enumerate,float,indentfirst}

\usepackage{tikz-qtree}        % regular trees (e.g. GB style)
\usepackage{tikz-dependency}   % dependency trees (bracket style)
\usepackage{gb4e}              % numbered lists for linguistic examples

\graphicspath{{images/}}

\usetheme{Rochester}
\usecolortheme{seagull}

\setbeamertemplate{footline}{\scriptsize{\hspace*{0.4cm}\insertframenumber}\vspace*{0.3cm}}
\beamertemplatenavigationsymbolsempty

\errorcontextlines 10000

\begin{document}

\title{\large{\sc Формальный синтаксис}}
\author{Константин Соколов}
%\institute[]{СПбГПУ}
%\date{Санкт-Петербург, 2015} 

\begin{frame}
    \thispagestyle{empty}
    \titlepage
\end{frame}

\begin{frame}{Литература к курсу (I)}
\textbf{Основная литература}
\bigskip
\begin{itemize}
	\item А. Е. Пентус, М. Р. Пентус. Теория формальных языков. (Учебное пособие). 2004.
	\item И. Л. Тимофеева. Математическая логика. (Курс лекций), 2007.
	\item Laura Kallmeyer. Parsing Beyond Context-Free Grammars. 2010.
	\item N. Francez, Sh. Wintner. Unification Grammars. 2011.
	\item R. Moot, Ch. Retore. The Logic of Categorial Grammars. 2012.
\end{itemize}
\end{frame}

\begin{frame}{Литература к курсу (II)}
\textbf{Дополнительная литература}
\bigskip
\begin{itemize}
	\item А. Ахо и др. Компиляторы. Принципы, технологии и инструментарий. М., Вильямс, 2016.
	\item Э. Бах. Неформальные лекции по формальной семантике. 2010
\end{itemize}
\end{frame}


\begin{frame}{}
\begin{center}
	\textbf{Модели синтаксиса}
\end{center}
\end{frame}

\begin{frame}{Предварительные замечания}
Необходимо различать
\bigskip
\begin{itemize}
	\item моделируемый объект (фрагмент английского языка)
	\item модель (X'-теория)
	\item аппарат моделирования (КС-грамматика)
\end{itemize}
\end{frame}

\begin{frame}{Формальный язык}
\begin{itemize}
	\item $\Sigma$ -- конечный алфавит
	\item $\Sigma^{*}$ -- множество строк (слов) над $\Sigma$
	\item $L \subseteq \Sigma^{*}$ -- (формальный) язык
\end{itemize}
\bigskip
Например, $\Sigma = \{a\}$, $\Sigma^{*} = \{\epsilon, a, aa, aaa, \ldots\}$, $L = a^{2n},$ где $n \in \mathbb{N}$
\end{frame}

\begin{frame}{Общая классификация}
\begin{itemize}
	\item регулярные, контекстно-свободные, контекстно-зависимые и рекурсивно-перечислимые языки
    \item языки типа 3, 2, 1, 0 (иерархия Хомского)
\end{itemize}
\bigskip
\begin{itemize}
    \item грамматики зависимостей и грамматики составляющих
    \item трансформационные и нетрансформационные грамматики
    \item лексикализованные и нелексикализованные грамматики
\end{itemize}
\end{frame}

\begin{frame}{}
\begin{center}
	\textbf{Грамматика зависимостей}
\end{center}
\end{frame}

% модель предложена Л. Теньером в книге "Основы структурного синтаксиса"
\begin{frame}{Структуры зависимостей -- I}
\begin{dependency}[theme = simple]
   \begin{deptext}[column sep=1em]
      A \& hearing \& is \& scheduled \& on \& the \& issue \& today \& . \\
   \end{deptext}
   \deproot{3}{ROOT}
   \depedge{2}{1}{ATT}
   \depedge[edge start x offset=-6pt]{2}{5}{ATT}
   \depedge{3}{2}{SBJ}
   \depedge{3}{9}{PU}
   \depedge{3}{4}{VC}
   \depedge{4}{8}{TMP}
   \depedge{5}{7}{PC}
   \depedge[arc angle=50]{7}{6}{ATT}
\end{dependency}
\end{frame}

\begin{frame}{Структуры зависимостей -- II}
\begin{dependency}[theme = simple]
   \begin{deptext}[column sep=1em]
      Сегодня \& мы \& начинаем \& заниматься \& математикой. \\
   \end{deptext}
   \depedge{3}{1}{}
   \depedge{3}{2}{}
   \depedge[edge start x offset=-8pt]{3}{4}{}
   \depedge[edge start x offset=-2pt]{4}{5}{}
\end{dependency}

\begin{dependency}[theme = simple]
   \begin{deptext}[column sep=1em]
      Мы \& сегодня \& начинаем \& заниматься \& математикой. \\
   \end{deptext}
   \depedge{3}{1}{}
   \depedge{3}{2}{}
   \depedge[edge start x offset=-8pt]{3}{4}{}
   \depedge[edge start x offset=-2pt]{4}{5}{}
\end{dependency}
\end{frame}

\begin{frame}{Структуры зависимостей -- III}
\begin{center}
\begin{dependency}[theme = simple]
   \begin{deptext}[column sep=1em]
      X \& Y \& Z \\
   \end{deptext}
   \depedge[edge start x offset=-5pt]{1}{2}{}
   \depedge[edge start x offset=-3pt]{2}{3}{}
\end{dependency}
\end{center}
\medskip
\begin{itemize}
    \item $X$ подчиняет $Y$, $Y$ зависит от $X$
    \item $X$ -- вершина, хозяин, ядро, главное слово
    \item $Y$ -- зависимое слово, слуга
    \item $Y$ непосредственно зависит от $X$, $Z$ -- опосредованно
\end{itemize}
\end{frame}

\begin{frame}{Структуры зависимостей -- IV}
Структура зависимостей большинства предложений естетвенного языка -- дерево\\
\medskip
\begin{itemize}
    \item принцип единственности корневого узла
    \item принцип единственности вершины
    \item принцип запрета на контур (цикл)
\end{itemize}
\end{frame}

\begin{frame}{Синтаксическая валентность словоформы}
\begin{itemize}
    \item активная -- способность присоединять зависимые
    \item пассивная -- способность выступать зависимым или корневым узлом
\end{itemize}
\end{frame}

\begin{frame}{Критерии выделения вершины -- I}
Критерий Базелла (эндоцентричности):\\
\medskip
{\small \textit{В предложении словоформа $W_1$ зависит от словоформы $W_2$, если пассивная валентность пары $W_1 + W_2$ в основном сопадает с пассивной валентностью $W_2$.}}

\begin{center}
\begin{dependency}[theme = simple]
   \begin{deptext}[column sep=1em]
      \textbf{очень} \& \textbf{высокая} \& гора \\
   \end{deptext}
   \depedge{2}{1}{}
   \depedge{3}{2}{}
\end{dependency}
\\
\begin{dependency}[theme = simple]
   \begin{deptext}[column sep=1em]
      высокая \& гора \\
   \end{deptext}
   \depedge{2}{1}{}
\end{dependency}
\begin{dependency}[theme = simple]
   \begin{deptext}[column sep=1em]
      очень \& гора \\
   \end{deptext}
   \depedge{2}{1}{?}
\end{dependency}
\end{center}
\end{frame}

\begin{frame}{Критерии выделения вершины -- II}
Критерий морфосинтаксического локуса:\\
\medskip
{\small \textit{Один из членов группы выступает в качестве представителя группы во внешних связях (локальной вершины группы) и подчиняет остальные члены группы.}}

\begin{center}
\begin{dependency}[theme = simple]
   \begin{deptext}[column sep=1em]
      \textbf{учитель} \& \textbf{сестры} \& пришел \\
   \end{deptext}
   \depedge{1}{2}{}
   \depedge{3}{1}{}
\end{dependency}
\begin{dependency}[theme = simple]
   \begin{deptext}[column sep=1em]
      *\textbf{учитель} \& \textbf{сестры} \& пришла \\
   \end{deptext}
   \depedge{2}{1}{?}
   \depedge{3}{2}{?}
\end{dependency}
\end{center}
\end{frame}

\begin{frame}{Корневая вершина -- I}
Где вершина в паре ``подлежащее -- сказуемое''?\\
\smallskip
\begin{itemize}
    \item сказуемое управляет падежом подлежащего
    \item сказуемое согласуется с подлежащим
\end{itemize}
\end{frame}

\begin{frame}{Корневая вершина -- II}
Критерий эндоцентричности: вершина -- сказуемое\\
\smallskip
\begin{center}
\begin{dependency}[theme = simple]
   \begin{deptext}[column sep=1em]
      что \& работает \\
   \end{deptext}
   \depedge{1}{2}{}
\end{dependency}
\begin{dependency}[theme = simple]
   \begin{deptext}[column sep=1em]
      что \& Иван \& работает \\
   \end{deptext}
   \depedge{1}{3}{}
   \depedge{3}{2}{}
\end{dependency}
\begin{dependency}[theme = simple]
   \begin{deptext}[column sep=1em]
      что \& Иван  \\
   \end{deptext}
   \depedge{1}{2}{?}
\end{dependency}
\end{center}
\end{frame}

\begin{frame}{Корневая вершина -- III}
Критерий морфосинтаксического локуса: вершина -- сказуемое\\
\smallskip
\begin{center}
\begin{dependency}[theme = simple]
   \begin{deptext}[column sep=1em]
      I \& knew \& that \& \textbf{he} \& \textbf{was} \& your \& friend \\
   \end{deptext}
   \depedge{2}{3}{}
   \depedge{3}{5}{}
   \depedge{5}{4}{}
\end{dependency}
\\
\begin{dependency}[theme = simple]
   \begin{deptext}[column sep=1em]
      *I \& knew \& that \& \textbf{he} \& \textbf{is} \& your \& friend \\
   \end{deptext}
   \depedge{2}{3}{}
   \depedge{3}{4}{?}
   \depedge{4}{5}{?}
\end{dependency}
\end{center}
\end{frame}

\begin{frame}{Виды синтаксических отношений (СинТагРус) -- I}
Cинтаксические отношения подразделяются на четыре группы:\\
\medskip
\begin{itemize}
    \item актантные 
    \item атрибутивные
    \item сочинительные
    \item служебные
\end{itemize}
\end{frame}

\begin{frame}{Виды синтаксических отношений (СинТагРус) -- II}
Актантные СинтО (пример):\\
\medskip
\begin{footnotesize}
\begin{itemize}
    \item \textit{предикативное}: сказуемое [X] $\to$ подлежащее [Y]
        \begin{itemize}
            \item[] \textit{Москва$_Y$ — столица$_X$ России}
            \item[] \textit{Петя$_Y$ какой-то странный$_X$}
            \item[] \textit{Иван$_Y$ все еще там$_X$}
            \item[] \textit{Сахару$_Y$ хватит$_X$ на всех}
            \item[] \textit{Любопытно$_X$, куда он пошел$_Y$}
        \end{itemize}
    \item \textit{дательно-субъектное}: состояние [X] $\to$ сущ., д.п. [Y]
        \begin{itemize}
            \item[] \textit{Мне$_Y$ можно$_X$ уйти?}
        \end{itemize}         
    \item \textit{\{1,2,3,4,5\}-комплетивное}: 1-ое комплетивное СинтО связывает слово с его вторым актантом и т.д.
        \begin{itemize}
            \item[] \textit{Фермер арендовал$_X$ гектар$_{Y_1}$ земли у$_{Y_2}$ совхоза за$_{Y_3}$ 50000 руб. на$_{Y_4}$ пять лет}
        \end{itemize}         
\end{itemize}
\end{footnotesize}
\end{frame}

\begin{frame}{Синтаксическая омонимия}
\begin{center}
\begin{dependency}[theme = simple]
   \begin{deptext}[column sep=1em]
      отзыв \& оппонента \& Петрова \\
   \end{deptext}
   \depedge{1}{2}{посессивное}
   \depedge{2}{3}{посессивное}
\end{dependency}
\\
\begin{dependency}[theme = simple]
   \begin{deptext}[column sep=1em]
      отзыв \& оппонента \& Петрова \\
   \end{deptext}
   \depedge{1}{2}{посессивное}
   \depedge{2}{3}{аппозитивное}
\end{dependency}
\end{center}
\end{frame}

\begin{frame}{Семантическая валентность лексемы -- I}
\textbf{Семантической валентностью (партиципантом)} лексемы $L$ называется любая переменная $X$, входящая в \textit{толкование} $L$.\\ \medskip
Лексема, имеющая один или несколько партиципантов, называется \textbf{предикатным словом (предикатом)}.\\ \medskip
Семантические \textbf{актанты} -- выражения, заполняющие валентности $L$ в предложении.
\end{frame}

\begin{frame}{Семантическая валентность лексемы -- II}
Принцип Кинэна:\\
\medskip
{\small \textit{Контролёр согласования является партиципантом мишени. \\Мишень управления является партиципантом контролёра.}}

\begin{center}
\begin{dependency}[theme = simple]
   \begin{deptext}[column sep=1em, row sep=1ex]
      большой-ИМ.П. \& дом-ИМ.П. \\
      \textit{pred} \& \textit{arg} \\
   \end{deptext}
   \depedge{2}{1}{}
\end{dependency}
\\
\begin{dependency}[theme = simple]
   \begin{deptext}[column sep=1em, row sep=1ex]
      вижу \& дом-ВИН.П. \\
      \textit{pred} \& \textit{arg} \\
   \end{deptext}
   \depedge{1}{2}{}
\end{dependency}
\end{center}
\end{frame}

\begin{frame}[fragile]{Stanford Dependency Parser}
\smallskip
{\small \texttt{Bell, a company which is based in LA, makes and distributes computer products.}}
\medskip
\begin{footnotesize}
\begin{verbatim}
nsubj(makes-11, Bell-1)
det(company-4, a-3)
appos(Bell-1, company-4)
nsubjpass(based-7, which-5)
auxpass(based-7, is-6)
rcmod(company-4, based-7)
prep(based-7, in-8)
pobj(in-8, LA-9)
root(ROOT-0, makes-11)
cc(makes-11, and-12)
conj(makes-11, distributes-13)
nn(products-15, computer-14)
dobj(makes-11, products-15)
\end{verbatim}
\end{footnotesize}
\end{frame}

\begin{frame}[fragile]{Link Grammar -- I}
\begin{scriptsize}
\begin{verbatim}
linkparser> If I can make it there, I'll make it anywhere.
Found 5 linkages (5 had no P.P. violations)
	Linkage 1, cost vector = (UNUSED=0 DIS=0 LEN=32)

    +--------------------------------Xp-------------------------------+
    +-------------------Wd------------------+                         |
    |      +--------------CO*s--------------+                         |
    |      +--------------Xc-------------+  |                         |
    |      |              +---MVp---+    |  |         +----MVp---+    |
    |      +Cs+Sp*i+---I--+-Os-+    |    |  +Sp*+--I--+-Os-+     |    |
    |      |  |    |      |    |    |    |  |   |     |    |     |    |
LEFT-WALL if I.p can.v make.v it there.r , I.p 'll make.v it anywhere . 
\end{verbatim}
\end{scriptsize}
\end{frame}

\begin{frame}[fragile]{Link Grammar -- II}
\begin{scriptsize}
\begin{verbatim}
linkparser> Пусть расцветают сто цветов.                           
Found 22 linkages (12 had no P.P. violations)
	Linkage 1, cost vector = (UNUSED=0 DIS=1 LEN=15)

    +---------------------------Xp---------------------------+
    +-----------Wd-----------+                               |
    |        +-------EI------+-----SIp3----+----Ngp---+      |
    |        |               |             |          |      |
LEFT-WALL пусть.p   расцветают.vnndn3p  сто.ci  цветов.ndmpg . 
\end{verbatim}
\end{scriptsize}
\end{frame}

\begin{frame}[fragile]{Link Grammar -- III}
\begin{scriptsize}
\begin{verbatim}
%
% Stems (roots)
%
% Total number of stems = 135775
% Number of stem connectors = 2997
% Number of stem classes = 6955
%
-голос.= -палатн.= :
  LLDLU+ or LLDLO+;

-зах.= -пл.= -полупери.= -пров.= :
  LLFLZ+;

-звенн.= -колонн.= -перстн.= -светн.= -цветн.= -шерстн.= :
  LLDLO+ or LLFUN+;

-кратн.= :
  LLDLO+ or LLDLU+;
\end{verbatim}
\end{scriptsize}
\end{frame}

\begin{frame}{}
\begin{center}
	\textbf{Грамматика непосредственных составляющих}
\end{center}
\end{frame}

\begin{frame}[fragile]{Грамматика составляющих -- I}
\begin{tiny}
\begin{center}
\Tree [.NP [.Adj золотые ] [.N пряди ] [.NP [.AP [.Adj склоняющейся ] [.PP [.Prep за ] [.NP [.Adj редкой ] [.N вещью ] ] ] ] [.N красавицы, ] [.AP [.Adj роющейся ] [.PP [.Prep меж ] [.NP [.N коробок ] ] ] ] ] ]
\end{center}
\end{tiny}
\end{frame}

\begin{frame}{Грамматика составляющих -- II}
\begin{small}
\begin{itemize}
    \item $S$ -- линейно упорядоченое множество слов
    \item размеченная система составляющих $C$ -- мн-во отрезков $S$, т.\,ч.
        \begin{itemize}
            \item $S \in C$
            \item если $A, B \in C$, то либо $A \subset B$, либо $B \subseteq A$, либо $A \cap B = \varnothing$
            \item $K$ -- множество категориальных символов
            \item каждому $c \in C$ приписан единственный $k \in K$
        \end{itemize}
    \item если $A \subset B$, то $B$ доминирует над $A$, $A$ -- составляющая $B$
\end{itemize}
\end{small}
\end{frame}

\begin{frame}{Грамматика составляющих -- III}
Структурные отношения:\\
\medskip
\begin{small}
\begin{itemize}
    \item если $A \subset B$, то $B$ доминирует над $A$, $A$ -- составляющая $B$
    \item если $A \subset B$ и не существует такого $C$, что $A \subset C \subset B$, \\то $A$ -- непосредственная составляющая $B$
\end{itemize}
\end{small}
\end{frame}



% Next -- выразительная сила синтаксического формализма
%       иерархия Хомского
%       Gaifman 1965
%       неадекватность CFG естественному языку; аргументация Хомского из "Синтаксических структур"; аргументация Шибера
%       пути построения выразительного формализма: трансформационный и нетрансформационный подход
%       тезис Джоши (1985), mild context sensitivity


\begin{frame}{}
    \thispagestyle{empty}
    \begin{center}
        {\large Спасибо!}
    \end{center}
\end{frame}


\end{document}

\documentclass{beamer}

\usepackage[T2A]{fontenc}
\usepackage[utf8]{inputenc}
\usepackage[english,russian]{babel}
\usepackage{amssymb,amsfonts,amsmath,mathtext}
\usepackage{cite,enumerate,float,indentfirst}

\usepackage{graphicx}
\usepackage{booktabs}
\usepackage{tabularx}

\usepackage{qtree}    % regular trees (e.g. GB style)
\usepackage{gb4e}     % numbered lists for linguistic examples

% CCG parse trees
\newcommand{\deriv}[2]
{  \renewcommand{\arraystretch}{.5}
$\begin{array}[t]{*{#1}{c}}
     #2
   \end{array}$ }
\newcommand{\gf}[1]{\textsf{\textsl{#1}}}
\newcommand{\cf}[1]{\mbox{\ensuremath{\cfont{#1}}}}
\newcommand{\uline}[1]
{\mc{#1}{\hrulefill} }
\newcommand{\mc}[2]
  {\multicolumn{#1}{c}{#2}}
\newcommand{\cfont}{\mathsf}
\newcommand{\bs}{\backslash}
\newcommand{\subsa}[1]{\hspace{-0.75mm}_{_{#1}}}
\newcommand{\subsb}[1]{\hspace{-0.10mm}_{_{#1}}}
\newcommand{\subs}[1]{\hspace{-0.40mm}_{#1}}
\newcommand{\subsf}[1]{\hspace{-0.75mm}_{_{#1}}}
\newcommand{\supsa}[1]{\hspace{-1.75mm}^{^{#1}} }
\newcommand{\supsb}[1]{\hspace{-0.80mm}^{^{#1}}  }
\newcommand{\sups}[1]{\hspace{-0.40mm}^{#1}}

\graphicspath{{images/}}

\usetheme{Rochester}
\usecolortheme{seagull}

\setbeamertemplate{footline}{\scriptsize{\hspace*{0.4cm}\insertframenumber}\vspace*{0.3cm}}
\beamertemplatenavigationsymbolsempty

\errorcontextlines 10000

\begin{document}

\title{\large{\sc Неоднозначность}}
\author{Константин Соколов}
\institute[]{СПбГПУ}
\date{Санкт-Петербург, 2015} 

\begin{frame}
    \thispagestyle{empty}
    \titlepage
\end{frame}


%%%%%%%%%%%%%%%%%%%%%
% 1. Повторение
%%%%%%%%%%%%%%%%%%%%%

\begin{frame}{}
\begin{center}
	\textbf{Повторение}
\end{center}
\end{frame}

\begin{frame}{Уровневая теория}
\setcounter{framenumber}{1}
\begin{itemize}
    \item какие уровни знаете
    \item что такое единица уровня
    \item приведите примеры единиц по уровням
    \item что такое эмическая vs. этическая единица
    \item единицы в словаре (словоформа, лексема, лемма)
    \item парадигматические, синтаксические и иерархические отношения
\end{itemize}
\end{frame}

\begin{frame}{Формальная модель}
\begin{itemize}
	\item критерии адекватности формальной модели
	\item формальная модель vs. формализм
	\item различия между естественными и искусственными языками
\end{itemize}
\end{frame}

\begin{frame}{Обсуждение}
Прокомментируйте следующие тезисы:\\
\medskip
\begin{itemize}
	\item Естественный язык не является регулярным.
	\item Термин ``слово'' запрещается. (Мельчук, КОМ, с. 98)
\end{itemize}
\end{frame}


%%%%%%%%%%%%%%%%%%%%%%%%
% 2. Неоднозначность
%%%%%%%%%%%%%%%%%%%%%%%%

\begin{frame}{}
\begin{center}
	\textbf{Неоднозначность}
\end{center}
\end{frame}

\begin{frame}{Предварительные замечания}
\begin{itemize}
	\item привычка к языку у человека, отсутствие таковой у компьютера
	\item человек замечает значительно меньше неоднозначности, чем компьютер 
	\item человек недооценивает сложность языка
	\item формальная модель может привносить сложность (accidental complexity)
\end{itemize}
\end{frame}

\begin{frame}{Омонимы}
\begin{itemize}
	\item полные омонимы
	    \begin{itemize}
	        \item \textit{ручка} (для письма) -- \textit{ручка} (маленькая рука)
	        \item \textit{наряд} (одежда) -- \textit{наряд} (распоряжение)
	    \end{itemize}
	\item частичные омонимы
	    \begin{itemize}
	        \item \textit{ласка} (зверь) -- \textit{ласка} (действие), но \textit{ласок} (р.п., мн.ч.), \textit{ласк} (р.п., мн.ч.)
	    \end{itemize}
	\item грамматические (совпадение отдельных форм)
	    \begin{itemize}
	        \item \textit{к трём} (часам), \textit{мы трём} (что-то)
	    \end{itemize}
\end{itemize}
\end{frame}

\begin{frame}{Омографы}
\begin{itemize}
	\item разные слова
	    \begin{itemize}
	        \item \textit{в\'{о}рон} (м.р., ед.ч., и.п.) -- \textit{вор\'{о}н} (ж.р., мн. ч., р.п.)
	        \item \textit{м\'{е}сти} (сущ., ж.р., р.п. или п.п.) -- \textit{мест\'{и}} (гл.)
	    \end{itemize}
	\item формы одного слова
	    \begin{itemize}
	        \item \textit{б\'{у}дите} -- \textit{буд\'{и}те}
	        \item \textit{м\'{е}ста} — \textit{мест\'{а}}
	    \end{itemize}
\end{itemize}
\end{frame}

\begin{frame}{``Паразитная'' омонимия}
\begin{itemize}
	\item ``\textit{Махабхарата} получает несколько гипотетических разборов, в том числе от псевдолексем \textit{махабхаронок}, \textit{махабхарать}.'' (с сайта НКРЯ)
	\item singularia tantum
	    \begin{itemize}
	        \item \textit{еда} (sg.t.), \textit{*\'{е}ды} (мн.ч., и.п.), но \textit{ед\'{ы}} (ед.ч., р.п.)
	    \end{itemize}
	\item pluralia tantum
	    \begin{itemize}
	        \item \textit{часы} (pl.t., предмет), ср. \textit{час} (единица времени), мн.ч. \textit{часы}
	        \item \textit{гр\'{я}зи} (pl.t., курорт), ср. \textit{грязь} (пачкающая субстанция), р.п. \textit{грязи}
	    \end{itemize}
\end{itemize}
\end{frame}

\begin{frame}{Слова с различной дистрибуцией}
\begin{itemize}
	\item \textit{disputare} (ит.) 
	    \begin{itemize}
	        \item обсуждать, спорить (disputare una questione -- обсуждать вопрос)
	        \item оспаривать (disputare la coppa -- бороться за кубок)
	    \end{itemize}
	\item \textit{сидеть} (в тюрьме) -- \textit{сидеть} (на зарплате)
	\item разные части речи
	    \begin{itemize}
	        \item \textit{water}, \textit{to water}
	        \item ``Мой дядя самых честных правил, когда не в шутку занемог; [\dots]'' -- \textit{(пунктуация по изданию 1825 г.)}
	    \end{itemize}
\end{itemize}
\end{frame}

\begin{frame}[fragile]{Cинтаксическая неоднозначность (I)}
\begin{center}
\Tree [.S [.NP [.Det The ] [.N boy ] ] [.VP [.V saw ] [.NP [.Det the ] [.N man ] [.PP [.P with ] [.NP [.Det the ] [.N telescope ] ] ] ] ] ]
\end{center}
\end{frame}

\begin{frame}[fragile]{Cинтаксическая неоднозначность (II)}
\begin{center}
\Tree [.S [.NP [.Det The ] [.N boy ] ] [.VP [.V saw ] [.NP [.Det the ] [.N man ] ] [.PP [.P with ] [.NP [.Det the ] [.N telescope ] ] ] ] ]
\end{center}
\end{frame}

\begin{frame}{Cинтаксическая неоднозначность (III)}
\begin{center}
\deriv{6}{
\gf{включи} & \gf{лампу} & \gf{и} & \gf{подсветку} & \gf{на} & \gf{кухне} \\
\uline{1} & \uline{1} & \uline{1} & \uline{1} & \uline{1} & \uline{1} \\
\cf{s\bs \supsa{-} np/ np} & \cf{np} & \cf{n/ np\bs \subsb{*} np} & \cf{n} & \cf{pp/ \subsa{\diamond} np} & \cf{np} \\
& \mc{2} {\hrulefill_{<}} \\
& \mc{2}{\cf{n/ np}} \\
&&&& \mc{2} {\hrulefill_{>}} \\
&&&& \mc{2}{\cf{pp}} \\
&&&& \mc{2} {\hrulefill_{t\mathbf{ypechange-6}}}\\
&&&& \mc{2}{\cf{n\bs \subsb{*} n}} \\
&&& \mc{3} {\hrulefill_{<}} \\
&&& \mc{3}{\cf{n}} \\
&&& \mc{3} {\hrulefill_{t\mathbf{ypechange-3}}}\\
&&& \mc{3}{\cf{np}} \\
& \mc{5} {\hrulefill_{>}} \\
& \mc{5}{\cf{n}} \\
& \mc{5} {\hrulefill_{t\mathbf{ypechange-4}}}\\
& \mc{5}{\cf{np}} \\
 \mc{6} {\hrulefill_{>}} \\
 \mc{6}{\cf{s\bs \supsa{-} np}} \\
}
\end{center}
\end{frame}

\begin{frame}{Cинтаксическая неоднозначность (III)}
\begin{center}
\deriv{6}{
\gf{включи} & \gf{лампу} & \gf{и} & \gf{подсветку} & \gf{на} & \gf{кухне} \\
\uline{1} & \uline{1} & \uline{1} & \uline{1} & \uline{1} & \uline{1} \\
\cf{s\bs \supsa{-} np/ np} & \cf{np} & \cf{n/ np\bs \subsb{*} np} & \cf{np} & \cf{pp/ \subsa{\diamond} np} & \cf{np} \\
& \mc{2} {\hrulefill_{<}} \\
& \mc{2}{\cf{n/ np}} \\
&&&& \mc{2} {\hrulefill_{>}} \\
&&&& \mc{2}{\cf{pp}} \\
& \mc{3} {\hrulefill_{>}} \\
& \mc{3}{\cf{n}} \\
&&&& \mc{2} {\hrulefill_{t\mathbf{ypechange-6}}}\\
&&&& \mc{2}{\cf{n\bs \subsb{*} n}} \\
& \mc{5} {\hrulefill_{<}} \\
& \mc{5}{\cf{n}} \\
& \mc{5} {\hrulefill_{t\mathbf{ypechange-4}}}\\
& \mc{5}{\cf{np}} \\
 \mc{6} {\hrulefill_{>}} \\
 \mc{6}{\cf{s\bs \supsa{-} np}} \\
}
\end{center}
\end{frame}

\begin{frame}{Анафора}
\begin{itemize}
	\item[] \textit{Гражданка Иванова}_i попала в \textit{автомобильную аварию}_j, \textit{которая}_{i/j} произошла сегодня утром на улице Ленина. После проведения экспертизы сотрудники ГИБДД признали \textit{её}_{i/j} виновником ДТП. 
\end{itemize}
\end{frame}

% область действия кванторов: every man admires a woman (каждый мужчина восхищается какой-то женщиной, все мужчины восхищаются одной и той же женщиной); каждый получит награду, которую заслуживает
\begin{frame}{Область действия кванторов}
\begin{itemize}
	\item Every man admires a woman
	    \begin{itemize}
	        \item $\forall x (man(x) \wedge (\exists y . woman(y) \wedge admires(x,y)))$ 
	        \item $\exists y (woman(y) \wedge \forall x (man(x) \wedge admires(x,y)))$ 
	    \end{itemize}
\end{itemize}
\end{frame}

\begin{frame}{Чтения \textit{de re} и \textit{de dicto}}
\smallskip
{\footnotesize	
\begin{exe}
	\ex de re + резумптивность
		\gll Dani yimca et ha-i\v{s}a \v{s}e hu mexapes \textit{ota}\\
             Дани найдёт ACC DET-женщину которая он ищет её-ACC\\
		\glt Дани найдёт женщину, которую ищет
	\ex de dicto + резумптивность
	    \gll ha-i\v{s}a \v{s}e Dani mexapes \textit{ota} crixa lihyot blondit\\
 	         DET-женщина которая Дани ищет её-ACC должна быть блондинкой\\
	    \glt Женщина, которую ищет Дани, должна быть блондинкой
\end{exe}		    
}
\end{frame}

% неоднозначный дейксис: Где привратник? -- гневно кричала она. -- Почему никто не подходит к двери?\\ -- К какой двери? -- спросил Лягушонок. [...]\\ -- К этой, конечно! (Пример Падучевой из Алисы в стране чудес)
% de re vs. de dicto: John seeks a unicorn; примеры про резумпцию в иврите; 
% неоднозначность вследствие эллипсиса: John loves his mother and Bill does too.
\begin{frame}{Неоднозначный дейксис}
Коммуникативная неудача (пример Е.~В.~Падучевой):\\
\bigskip
\begin{itemize}
	\item[] \textit{-- Где привратник? -- гневно кричала она. -- Почему никто не подходит к двери?}
	\item[] \textit{-- К какой двери? -- спросил Лягушонок. [\dots]}
	\item[] \textit{-- К этой, конечно!}
\end{itemize}
\end{frame}

\begin{frame}[fragile]{Spurious ambiguity}
\begin{tiny}
\begin{verbatim}
John        loves            Mary                 John         loves          Mary
np          (s\np)/np        np                   np           (s\np)/np      np
--------T>  --------------------->                             ----------------------->
s/(s\np)              s\np                                              s\np
---------------------------->                     -----------------------------<
             s                                                  s
 
John        loves            Mary                 John         loves          Mary 
np          (s\np)/np        np                   np           (s\np)/np     np
--------T>                   ---------T>          ---------T>
s/(s\np)                     s/(s\np)             s/(s\np)
            --------------------------B<          -----------------------B>
                      s\np                                   s/np
----------------------------->                              --------------------------->
             s                                                  s

John        loves            Mary                John           loves         Mary
np          (s\np)/np        np                  np             (s\np)/np     np
                             ---------T>         --------T>                   --------T<
                             s/(s\np)            s/(s\np)                     s\(s/np)
            --------------------------B<         ----------------------B>
                      s\np                                   s/np
-----------------------------<                              ---------------------------<
             s                                                  s
\end{verbatim}
\end{tiny}
\end{frame}



%%%%%%%%%%%%%%%%
% 3. Метаязык
%%%%%%%%%%%%%%%%

\iffalse

\begin{frame}{}
\begin{center}
	\textbf{Метаязык}
\end{center}
\end{frame}

\begin{frame}{Словарное толкование}
Толково-комбинаторный словарь \\(Мельчук, Жолковский и др., 1984)\\
\smallskip
\begin{center}
	\begin{figure}[H]
		\includegraphics[scale=0.4]{tks.png} 
	\end{figure}
\end{center}	
\end{frame}

\begin{frame}{Метаязык в логике}
\begin{itemize}
	\item ``Иногда мы хотим говорить о каком-либо языке $L_1$ на каком-нибудь другом языке $L_2$. В таком случае $L_1$ обычно называют \textit{языком-объектом} (или \textit{предметным языком}), а $L_2$ -- \textit{метаязыком}.'' (Х. Карри)
	\item Р. Монтегю и метод трансляции
	\item Г.~С.~Цейтин о естественном языке
\end{itemize}
\end{frame}

\fi

% 4. Устранение неоднозначности с помощью обучения с учителем (supervised learning).
\begin{frame}{}
\begin{center}
	\textbf{Корпусная лингвистика}
\end{center}
\end{frame}

\begin{frame}{Корпуса (I)}
\begin{itemize}
    \item задача сбора текстов (подкорпуса)
    \item задача разметки корпуса
    \item задача поиска в корпусе
    \item задача обучения по корпусу
\end{itemize}
\end{frame}

\begin{frame}{Корпуса (II)}
\begin{center}
Национальный корпус русского языка, \texttt{http://ruscorpora.ru}\\
Открытый корпус, \texttt{http://opencorpora.org}
\end{center}
\end{frame}

\begin{frame}{Подкорпуса в НКРЯ}
\begin{itemize}
	\item основной
	\item синтаксический
	\item газетный
	\item параллельный
	\item диалектный
	\item поэтический
	\item устный
	\item акцентологический
	\item исторический и др.
\end{itemize}
\end{frame}

\begin{frame}{Поиск в НКРЯ (I)}
\begin{center}
	\begin{figure}[H]
		\includegraphics[scale=0.35]{ruscorpora-search.png} 
	\end{figure}
\end{center}	
\end{frame}

\begin{frame}{Поиск в НКРЯ (II)}
\begin{center}
	\begin{figure}[H]
		\includegraphics[scale=0.3]{ruscorpora-result.png} 
	\end{figure}
\end{center}	
\end{frame}

\begin{frame}{Морфологическая разметка в НКРЯ (I)}
\begin{itemize}
    \item набор помет (tagset) -- отражение принятой формальной модели
	\item НКРЯ опирается на морфологическую модель, представленную в <<Грамматическом словаре русского языка>> А. А. Зализняка (М., 1977; 4-е изд., М., 2003).
	\item разметка и снятие неоднозначности возможны только относительно принятой модели
\end{itemize}
\end{frame}

\begin{frame}{Морфологическая разметка в НКРЯ (II)}
\begin{center}
	\begin{figure}[H]
		\includegraphics[scale=0.55]{ruscorpora-cases.png} 
	\end{figure}
\end{center}	
\end{frame}

\begin{frame}{Морфологическая разметка в НКРЯ (III)}
\begin{center}
	\begin{figure}[H]
		\includegraphics[scale=0.35]{ruscorpora-grm.png} 
	\end{figure}
\end{center}	
\end{frame}

\begin{frame}{Синтаксическая разметка в НКРЯ (I)}
\begin{itemize}
	\item СинТагРус -- синтаксически размеченный подкорпус НКРЯ
	\item опирается на МСТ И.~А.~Мельчука и А.~К.~Жолковского
	    \begin{itemize}
	        \item синтаксическая структура -- дерево зависимостей
	        \item в узлах -- слова
	        \item ветви помечены именами синтаксических отношений
	    \end{itemize}
\end{itemize}
\end{frame}

\begin{frame}{Синтаксическая разметка в НКРЯ (II)}
\begin{center}
	\begin{figure}[H]
		\includegraphics[scale=0.3]{ruscorpora-syntactic-relations.png} 
	\end{figure}
\end{center}	
\end{frame}

\begin{frame}{Семантическая разметка в НКРЯ}
\begin{center}
	\begin{figure}[H]
		\includegraphics[scale=0.32]{ruscorpora-semantic-tags.png} 
	\end{figure}
\end{center}
\end{frame}

\begin{frame}{Семантическая разметка в GMB (I)}
\begin{itemize}
	\item Groningen Meaning Bank, \texttt{http://gmb.let.rug.nl}
	\item синтаксическая модель -- комбинаторная категориальная грамматика (CCG)
	\item семантическая модель -- теория репрезентации дискурса (DRT)
\end{itemize}
\end{frame}

\begin{frame}{Семантическая разметка в GMB (II)}
\begin{center}
	\begin{figure}[H]
		\includegraphics[scale=0.45]{gmb-pos-roles.png} 
	\end{figure}
	\begin{figure}[H]
		\includegraphics[scale=0.3]{gmb-synsem.png} 
	\end{figure}
\end{center}
\end{frame}

\begin{frame}{Семантическая разметка в GMB (III)}
\begin{center}
	\begin{figure}[H]
		\includegraphics[scale=0.6]{gmb-drs.png} 
	\end{figure}
\end{center}
\end{frame}

\begin{frame}{OpenCorpora (I)}
\begin{center}
	\begin{figure}[H]
		\includegraphics[scale=0.12]{yes-we-can.jpg} 
	\end{figure}
\end{center}
\end{frame}

\begin{frame}{OpenCorpora (II)}
\begin{center}
	\begin{figure}[H]
		\includegraphics[scale=0.35]{opencorpora-pools.png} 
	\end{figure}
\end{center}
\end{frame}

\begin{frame}{OpenCorpora (III)}
\begin{center}
	\begin{figure}[H]
		\includegraphics[scale=0.4]{opencorpora-task.png} 
	\end{figure}
\end{center}
\end{frame}

\begin{frame}{OpenCorpora (IV)}
\begin{center}
	\begin{figure}[H]
		\includegraphics[scale=0.4]{opencorpora-average.png} 
	\end{figure}
\end{center}
\end{frame}

\begin{frame}{OpenCorpora (V)}
\begin{center}
	\begin{figure}[H]
		\includegraphics[scale=0.41]{opencorpora-percent.png} 
	\end{figure}
\end{center}
\end{frame}


\begin{frame}{}
\begin{center}
	\textbf{Домашнее задание}
\end{center}
\end{frame}

\begin{frame}{}
\begin{center}
Сделать 100 заданий на OpenCorpora из разных пулов.\\
\medskip
Внимательно читайте инструкции!\\
\medskip
\texttt{http://opencorpora.org}
\end{center}
\end{frame}

\begin{frame}{}
    \thispagestyle{empty}
    \begin{center}
        {\large Спасибо!}
    \end{center}
\end{frame}


\end{document}
